\documentclass[a4paper,11pt]{article}
\usepackage{setspace}
\usepackage{graphicx}
\usepackage{geometry}
\usepackage{fancyhdr}
\usepackage{amsmath}
\usepackage{xeCJK}
\usepackage{amssymb}
\usepackage{listings}
\usepackage{xcolor}
\usepackage{fontspec}
\usepackage{enumerate}

\setmonofont{Courier New}
\setCJKmainfont[BoldFont={Adobe Heiti Std}, ItalicFont={Adobe Kaiti Std}]{Adobe Song Std}
\setCJKmonofont{Courier New}
\linespread{1.3}
\setlength{\parindent}{2em}%首行  缩进
\geometry{left=2.5cm,right=2.5cm,top=2.5cm,bottom=2.5cm}
\addtolength{\parskip}{2pt} % 设置段间距
\begin{document}
\begin{center}
	\huge{\textbf{实验2 解线性方程组的迭代法\\}}
\end{center}
\begin{flushright}
	201311211914\\
	赵帅帅\\
	2016--3--29\\
\end{flushright}
\begin{Large}
	\textbf{一、 实验问题}
\end{Large}

	给定下列几个不同类型的线性方程组,请分别采用\textbf{Jacobi}迭代法,\textbf{Gauss-Seidel}迭代法和\textbf{SOR}迭代法求解。
\begin{enumerate}
	\item 线性方程组
	\begin{displaymath} %开始数学环境
		\left[
		\begin{array}{cccccccccc}
			4 & 2 & -3 & -1 & 2 & 1 & 0 & 0 & 0 & 0\\
		    8 & 6 & -5 & -3 & 6 & 5 & 0 & 1 & 0 & 0\\
		    4 & 2 & -2 & -1 & 3 & 2 & -1 & 0 & 3 & 1\\
		    0 & -2 & 1 & 5 & -1 & 3 & -1 & 1 & 9 & 4\\
		    -4 & 2 & 6 & -1 & 6 & 7 & -3 & 3 & 2 & 3\\
		    8 & 6 & -8 & 5 & 7 & 17 & 2 & 6 & -3 & 5\\
		    0 & 2 & -1 & 3 & -4 & 2 & 5 & 3 & 0 & 1\\
		    16 & 10 & -11 & -9 & 17 & 34 & 2 & -1 & 2 & 2\\
		    4 & 6 & 2 & -7 & 13 & 9 & 2 & 0 & 12 & 4\\
		    0 & 0 & -1 & 8 & -3 & -24 & -8 & 6 & 3 & -1
		\end{array}
		\right]
		\left[
		\begin{array}{c}
			x_1\\x_2\\x_3\\x_4\\x_5\\x_6\\x_7\\x_8\\x_9\\x_{10}
		\end{array}
		\right]
		=
		\left[
		\begin{array}{c}
			5\\ 12\\ 3\\ 2\\ 3\\ 46\\ 13\\ 38\\ 19\\ -21
		\end{array}
		\right]
	\end{displaymath}
	精确解$x^{*}=\left(1,-1,0,1,2,0,3,1,-1,2\right)^T$
	\item 对称正定方程组
	\begin{displaymath} %开始数学环境
		\left[
		\begin{array}{cccccccc}
			4 & 2 & -4 & 0 & 2 & 4 & 0 & 0\\
    		2 & 2 & -1 & -2 & 1 & 3 & 2 & 0\\
		    -4 & -1 & 14 & 1 & -8 & -3 & 5 & 6\\
		    0 & -2 & 1 & 6 & -1 & -4 & -3 & 3\\
		    2 & 1 & -8 & -1 & 22 & 4 & -10 & -3\\
		    4 & 3 & -3 & -4 & 4 & 11 & 1 & -4\\
		    0 & 2 & 5 & -3 & -10 & 1 & 14 & 2\\
		    0 & 0 & 6 & 3 & -3 & -4 & 2 & 19
		\end{array}
		\right]
		\left[
		\begin{array}{c}
			x_1\\x_2\\x_3\\x_4\\x_5\\x_6\\x_7\\x_8
		\end{array}
		\right]
		=
		\left[
		\begin{array}{c}
			0 \\ -6 \\ 6 \\ 23 \\ 11 \\ -22 \\ -15 \\ 45
		\end{array}
		\right]
	\end{displaymath}
	精确解$x^{*}=\left(1,-1,0,2,1,-1,0,2\right)^T$
	\item 三对角线方程组
	\begin{displaymath} %开始数学环境
		\left[
		\begin{array}{cccccccccc}
			4 & -1 & 0 & 0 & 0 & 0 & 0 & 0 & 0 & 0\\
		    -1 & 4 & -1 & 0 & 0 & 0 & 0 & 0 & 0 & 0\\
		    0 & -1 & 4 & -1 & 0 & 0 & 0 & 0 & 0 & 0\\
		    0 & 0 & -1 & 4 & -1 & 0 & 0 & 0 & 0 & 0\\
		    0 & 0 & 0 & -1 & 4 & -1 & 0 & 0 & 0 & 0\\
		    0 & 0 & 0 & 0 & -1 & 4 & -1 & 0 & 0 & 0\\
		    0 & 0 & 0 & 0 & 0 & -1 & 4 & -1 & 0 & 0\\
		    0 & 0 & 0 & 0 & 0 & 0 & -1 & 4 & -1 & 0\\
		    0 & 0 & 0 & 0 & 0 & 0 & 0 & -1 & 4 & -1\\
		    0 & 0 & 0 & 0 & 0 & 0 & 0 & 0 & -1 & 4
		\end{array}
		\right]
		\left[
		\begin{array}{c}
			x_1\\x_2\\x_3\\x_4\\x_5\\x_6\\x_7\\x_8\\x_9\\x_{10}
		\end{array}
		\right]
		=
		\left[
		\begin{array}{c}
			7\\ 5\\ -13\\ 2\\ 6\\ -12\\ 14\\ -4\\ 5\\ -5
		\end{array}
		\right]
	\end{displaymath}
	精确解$x^{*}=\left(2,1,-3,0,1,-2,3,0,1,-1\right)^T$
\end{enumerate}
\begin{Large}
	\textbf{二、 实验要求}
\end{Large}
\begin{enumerate}
	\item 应用迭代法求解线性方程组,并与直接法做比较。
	\item 分别对不同精度要求,如$\varepsilon=10^{-3},10^{-4},10^{-5}$,利用所需迭代次数体会该迭代法收敛快慢。
	\item 对方程做(2),(3)使用\textbf{SOR}方法时,选取松弛因子$\omega=0.8,0.9,1,1.1,1.2$等,试观察对算法收敛的影响,并找出你所选用松弛因子的最佳值。
	\item 编制出各种迭代法的程序并给出计算结果。
\end{enumerate}
\begin{Large}
	\textbf{三、 实验目的与意义}
\end{Large}
\begin{enumerate}
	\item 通过上机计算了解迭代法求解线性方程组的特点;掌握求解线性方程组的各类迭代算法。。
	\item 体会上机计算时,终止准则$\Big\|x^{(k+1)}-x^{k}\Big\|_\infty<\varepsilon$,对控制迭代解精度的有效性。
	\item 体会初始值$x^{(0)}$和松弛因子$\omega$的选取,对迭代收敛速度的影响。
\end{enumerate}
\begin{Large}
	\textbf{四、 算法}
\end{Large}
\begin{enumerate}
	\item \textbf{Jacobi迭代法\\}
	用\textbf{Jacobi}迭代法求解方程组$\boldsymbol{Ax=b}$,一维数组$\boldsymbol{x}^{(0)}$和$\boldsymbol{x}$分别存放迭代向量$\boldsymbol{x}^{(k)}$和$\boldsymbol{x}^{(k+1)}$。
	\begin{description}
		\item[输入 ] 矩阵$\boldsymbol{A}=\left(a_{ij}\right)$,右端项$\boldsymbol{b}=\left(b_1,b_2,\cdots,b_n\right)^T$,维数$n$,初始向量$\boldsymbol{x}^{(0)}$,精度要求$\varepsilon$,最大迭代次数$N$
		\item[输出 ] 迭代解$x_1,x_2,\cdots,x_n$和迭代次数$k$
	\end{description}
	\begin{enumerate}[(1)]
		\item 对于$k=1,2,\cdots,N$,可循环执行步2到步5
		\item 对于$i=1,2,\cdots,n$,计算
			\begin{displaymath}
				x_{i}\Leftarrow{}x_{i}^{(0)}+\frac{b_{i}-\sum\limits_{j=1}^{n}a_{ij}x_{j}^{(0)}}{a_{ii}}\footnote{算法原始表达式$x_{i}\Leftarrow{}\frac{b_{i}-\sum\limits_{j=1}^{i-1}a_{ij}x_{j}^{(k)}-\sum\limits_{j=i+1}^{n}a_{ij}x_{j}^{(k)}}{a_{ii}}$}
			\end{displaymath}
		\item 置$R=\max\limits_{1\leqslant{}i\leqslant{}n}\Big|x_{i}-x_{i}^{(0)}\Big|$
		\item 如果$R\leqslant{}\varepsilon$\\输出$x_1,x_2,\cdots,x_n,k$
		\item $x_{i}^{(0)}\Leftarrow{}x_{i},i=1,2,\cdots,n$
		\item 已达到最大迭代次数,输出$x_1,x_2,\cdots,x_n,k$
	\end{enumerate}
	\item \textbf{Gauss-Seidel迭代法\\}
	用\textbf{Gauss-Seidel}迭代法求解方程组$\boldsymbol{Ax=b}$,一维数组$\boldsymbol{x}^{(0)}$和$\boldsymbol{x}$分别存放迭代向量$\boldsymbol{x}^{(k)}$和$\boldsymbol{x}^{(k+1)}$。
	\begin{description}
		\item[输入 ] 矩阵$\boldsymbol{A}=\left(a_{ij}\right)$,右端项$\boldsymbol{b}=\left(b_1,b_2,\cdots,b_n\right)^T$,维数$n$,初始向量$\boldsymbol{x}^{(0)}$,精度要求$\varepsilon$,最大迭代次数$k$
		\item[输出 ] 迭代解$x_1,x_2,\cdots,x_n$和迭代次数$k$
	\end{description}
	\begin{enumerate}[(1)]
		\item 对于$k=1,2,\cdots,N$,可循环执行步2到步5
		\item 对于$i=1,2,\cdots,n$,计算
			\begin{displaymath}
				x_{i}\Leftarrow{}\frac{b_{i}-\sum\limits_{j=1}^{i-1}a_{ij}x_{j}^{(k+1)}-\sum\limits_{j=i+1}^{n}a_{ij}x_{j}^{(k)}}{a_{ii}}
			\end{displaymath}
		\item 置$R=\max\limits_{1\leqslant{}i\leqslant{}n}\Big|x_{i}-x_{i}^{(0)}\Big|$
		\item 如果$R\leqslant{}\varepsilon$\\输出$x_1,x_2,\cdots,x_n,k$
		\item $x_{i}^{(0)}\Leftarrow{}x_{i},i=1,2,\cdots,n$
		\item 已达到最大迭代次数,输出$x_1,x_2,\cdots,x_n,k$
	\end{enumerate}
	\item \textbf{SOR迭代法\\}
	用\textbf{SOR}方法($\omega=1$时为\textbf{GS}迭代法)求解方程组$\boldsymbol{Ax=b}$,计算公式为
	\begin{eqnarray}
		x_{i}^{k+1}&=&x_{i}^{k}+\Delta{}x_{i},i=1,2,\cdots,n \nonumber \\
		\Delta{}x_{i}&=&\frac{\omega}{a_{ii}}\left(b_{i}-\sum\limits_{j=1}^{i-1}a_{ij}x_{j}^{(k+1)}-\sum\limits_{j=i+1}^{n}a_{ij}x_{j}^{(k)}\right) \nonumber
	\end{eqnarray}
	用一维数组$\boldsymbol{x}$存放迭代向量$\boldsymbol{x}^{k}$,终止准则为
	\begin{displaymath}
		\max\limits_{1\leqslant{}i\leqslant{}n}\Big|x_{i}^{(k+1)}-x_{i}^{(k)}\Big|=\max\limits_{1\leqslant{}i\leqslant{}n}\Big|\Delta{}x_{i}\Big|\leqslant{}\varepsilon
	\end{displaymath}
	$\varepsilon$为设定的精度要求。为防止死循环,限制最大迭代次数为$N$。
	\begin{description}
		\item[输入 ] $\boldsymbol{A}=\left(a_{ij}\right),\boldsymbol{b}=\left(b_1,b_2,\cdots,b_n\right)^T,\boldsymbol{x}^{(0)}=\left(x_{1}^{(0)},x_{2}^{(0)},\cdots,x_{n}^{(0)}\right),\varepsilon,N,\omega$
		\item[输出 ] 迭代解$x_1,x_2,\cdots,x_n$和迭代次数$k$
	\end{description}
	\begin{enumerate}[(1)]
		\item $x_{i}\Leftarrow{}x_{i}^{(0)},i=1,2,\cdots,n$
		\item 对于$k=1,2,\cdots,N$,循环执行步3到步8
		\item $R\Leftarrow{}0$
		\item 对于$i=1,2,\cdots,n$,循环执行步5到步7
		\item $R_{1}\Leftarrow{}\Delta{}x_{i}=\frac{\omega}{a_{ii}}\left(b_{i}-\sum\limits_{j=1}^{n}a_{ij}x_{j}\right)$
		\item 如果$|R_1|>|R|$,则$R\Leftarrow{}R_1$
		\item $x_i\Leftarrow{}x_i+R_1$
		\item 如果$|R|\leqslant{}\varepsilon$,输出:$x_1,x_2,\cdots,x_n,k$
		\item 已达到最大迭代次数,输出:$x_1,x_2,\cdots,x_n,k$\\此时,可另选$x^{(0)}$或$\omega$重新迭代计算
	\end{enumerate}
\end{enumerate}
\begin{Large}
	\textbf{五、 实验代码}
\end{Large}

\lstset{
	backgroundcolor=\color[rgb]{0.9,0.9,0.9}, %设置背景颜色
    tabsize=4,    %tab大小
    language=[GNU]C++,
    % basicstyle=\ttfamily\small,	% 代码字体大小
    basicstyle=\small,	% 代码字体大小
    upquote=true,
    aboveskip={1.5\baselineskip},
    columns=fixed,
    extendedchars=false,
    breaklines=true,
    % prebreak = \raisebox{0ex}[0ex][0ex]{\ensuremath{\hookleftarrow}},
    frame=single, % 代码加边框
    rulesepcolor=\color{white}, % 边框颜色
    numbers=left, % 行号的位置
    stepnumber=1, % 两个行号之间的差值
    numberstyle=\scriptsize\color[rgb]{0.205, 0.142, 0.73},% 行号大小 颜色
    numbersep=5pt, % 行号与代码间的距离
    showtabs=false, % 显示tab
    showspaces=false, %显示空格
    showstringspaces=false, %显示字符串中的空格
    identifierstyle=\ttfamily, %
    keywordstyle=\bfseries\color{blue},% 关键字颜色
    morekeywords={}, %增加关键字
    commentstyle=\color[rgb]{0,0.5,0}, % 注释颜色
    stringstyle=\color[rgb]{0.627,0.126,0.941},% 字符串颜色
    escapeinside={\%*}{*)}, % if you want to add LaTeX within your code
    % title=\lstname % 显示用\lstinputlisting导入的文件名
}


\begin{enumerate}
	\item \textbf{Jacobi迭代法}
	\begin{spacing}{1.0}
	\begin{lstlisting}[language=C++]
bool Jacobi(double a[N][N], double b[N], double x[N], double e)
{
    double temp_x[N] = {0};
    for(int k = 1; k <= max_iteration; k ++)
    {
        double R = 0;
        for(int i = 0; i < N; i ++)
        {
            if(a[i][i] == 0)
            {
                cout<<"Divided by zero!"<<endl;
                return false;
            }
            // 迭代
            double sum = 0;
            for(int j = 0; j < N; j ++)
            {
                sum += a[i][j] * temp_x[j];
            }
            x[i] = temp_x[i] + (b[i] - sum) / a[i][i];
            if(fabs(x[i] - temp_x[i]) > R)
            {
                R = fabs(x[i] - temp_x[i]);
            }
            // 判断是否发散
            if(_isnan(x[i])!=0)
            {
                cout<<"算法不收敛!"<<endl;
                return false;
            }
        }
        //判断是否达到精度
        if(R <= e)
        {
            cout<<"迭代次数:"<<k<<endl;
            return true;
        }
        // 记录本次迭代结果
        for(int i = 0; i < N; i ++)
        {
            temp_x[i] = x[i];
        }
    }
    cout<<"迭代次数:"<<max_iteration<<endl;
    return true;
}
	\end{lstlisting}
	\item \textbf{Gauss-Seidel迭代法}
	\begin{lstlisting}
bool Gauss_Seidel(double a[N][N], double b[N], double x[N], double e)
{
	double temp_x[N] = {0};
    for(int k = 1; k <= max_iteration; k ++)
    {
        double R = 0;
        for(int i = 0; i < N; i ++)
        {
            if(a[i][i] == 0)
            {
                cout<<"Divided by zero!"<<endl;
                return false;
            }
			//迭代计算解
            double sum = 0;
            for(int j = 0; j < N; j ++)
            {
				//不加自己
				if(j == i)
				{
					continue;
				}
                sum += a[i][j] * x[j];
            }
            x[i] = (b[i] - sum) / a[i][i];
            if(fabs(x[i] - temp_x[i]) > R)
            {
                R = fabs(x[i] - temp_x[i]);
            }
            // 判断是否发散
            if(_isnan(x[i])!=0)
            {
                cout<<"算法不收敛!"<<endl;
                return false;
            }
        }
		// 判断是否达到精度
        if(R <= e)
        {
            cout<<"迭代次数:"<<k<<endl;
            return true;
        }
		//将上一次计算的结果放置到临时x中,为下次迭代做准备
        for(int i = 0; i < N; i ++)
        {
            temp_x[i] = x[i];
        }
    }
    cout<<"迭代次数:"<<max_iteration<<endl;
    return true;
}
	\end{lstlisting}
	\item \textbf{SOR迭代法:}
	\begin{lstlisting}
bool SOR(double a[N][N], double b[N], double x[N], double e, double w)
{
    for(int k = 1; k <= max_iteration; k ++)
    {
        double R = 0;
        for(int i = 0; i < N; i ++)
        {
            // 迭代
            double sum = 0;
            for(int j = 0; j < N; j ++)
            {
                sum += a[i][j] * x[j];
            }
            double R1 = w * (b[i] - sum) / a[i][i];
            if(fabs(R1) > fabs(R))
            {
                R = R1;
            }
            x[i] += R1;
            // 判断是否发散
            if(_isnan(x[i])!=0)
            {
                cout<<"算法不收敛!"<<endl;
                return false;
            }
        }
        // 判断是否达到精度
        if(fabs(R) <= e)
        {
            cout<<"迭代次数:"<<k<<endl;
            return true;
        }
    }
    cout<<"迭代次数:"<<max_iteration<<endl;
    return true;
}
	\end{lstlisting}
	\end{spacing}
\end{enumerate}
\begin{Large}
	\textbf{六、实验结果}
\end{Large}
\begin{quote}
	\begin{description}
		\item[操作系统:] Windows 10 企业版
		\item[GCC 版本:] 4.7.1
		\item[G++ 版本:] 4.7.1
	\end{description}
\end{quote}
\begin{enumerate}
	\item 直接法与迭代法比较
	\begin{enumerate}[(1)]
	\item 直接法在主对角线元素不为0的情况下能保证解出方程组,但是有些算法要求方程组必须满足一些特殊条件;迭代法不一定能保证求解方程组。
	\item 对于直接法,求解方程组的时间与矩阵的大小有关;迭代法的时间是不能确定的,即使对于同一方程组,不同的初值和精度会影响到方程组求解时间,对于SOR算法,收敛因子对于求解时间也有影响。
	\end{enumerate}
	\item 精度对收敛的影响\footnote{SOR算法收敛因子为1.1}
	\begin{enumerate}[(1)]
		\item 方程组1
			\begin{center}
			\begin{tabular}{|c|c|c|c|}	
			\hline
			算法 & $e=10^{-3}$ & $e=10^{-4}$ & $e=10^{-5}$ \\
			\hline
			Jacobi迭代法 & 不收敛 & 不收敛 & 不收敛\\
			\hline
			Gauss-Seidel迭代法 & 不收敛 & 不收敛 & 不收敛\\
			\hline
			SOR迭代法 & 不收敛 & 不收敛 & 不收敛\\
			\hline
			\end{tabular}
		\end{center}
		\item 方程组2
		\begin{center}
			\begin{tabular}{|c|c|c|c|}
			\hline
			算法 & $e=10^{-3}$ & $e=10^{-4}$ & $e=10^{-5}$ \\
			\hline
			Jacobi迭代法 & 不收敛 & 不收敛 & 不收敛\\
			\hline
			Gauss-Seidel迭代法 & 133 & 570 & 1006\\
			\hline
			SOR迭代法 & 197 & 554 & 991 \\
		 	\hline
			\end{tabular}
		\end{center}
		\item 方程组3
			\begin{center}
			\begin{tabular}{|c|c|c|c|}
			\hline
			算法 & $e=10^{-3}$ & $e=10^{-4}$ & $e=10^{-5}$ \\
			\hline
			Jacobi迭代法 & 11 & 14 & 18\\
			\hline
			Gauss-Seidel迭代法 & 7 & 9 & 11\\
			\hline
			SOR迭代法 & 8 & 9 & 11 \\
			\hline
			\end{tabular}
		\end{center}
	\end{enumerate}
	由上表知,精度要求越高,算法收敛越慢。
	\item SOR算法中收敛因子对算法收敛的影响\footnote{精度要求$10^{-5}$}
	\begin{center}
		\begin{tabular}{|c|c|c|c|c|c|}
		\hline
		方程组 & $\omega=0.8$ & $\omega=0.9$ & $\omega=1.0$ & $\omega=1.1$ & $\omega=1.2$ \\
		\hline
		方程组2 & 1083 & 1079 & 1006 & 911 & 809\\
		\hline
		方程组3 & 15 & 13 & 11 & 11 & 13\\
		\hline
		\end{tabular}
	\end{center}
	由上表知,对于方程组2,$\omega=1.2$时,算法收敛较快;对于方程组3,$\omega=1.0,1.1$时,算法收敛较快。
\end{enumerate}
\begin{Large}
	\textbf{七、实验感受}
\end{Large}
\begin{enumerate}
	\item 因为算法都有,所以编程没有什么难处,但是得到实验数据需要有耐心。
	\item 写完程序之后,感觉对于比较小的矩阵还是用直接法求解比较好。
\end{enumerate}
\end{document}